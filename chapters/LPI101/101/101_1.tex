\renewcommand{\sect}{101 Hardwareeinstellungen}

\cardfrontfoot{\chap/\sect}

\begin{flashcard}[Information]{Kernel-Version Identifizieren}
	\begin{itemize}
		\item Command: \texttt{uname -r}. Output: Kernel-Version
		\item \texttt{/lib/modules} \textrightarrow\ Kernel Historie
		\item \texttt{/usr/src/<KERNEL-VERSION>} \textrightarrow\ Source Code
	\end{itemize}
\end{flashcard}

\begin{flashcard}[Information]{Kernel-Version Codierung}
	\textbf{Beispiel}: 2.6.11.4-21.9-smp
	\begin{description}
		\item [1. Ziffer] Major release
		\item [2. Ziffer] Minor release
			\item Gerade Ziffern \textrightarrow\ Stable, Ungerade Ziffern \textrightarrow\ Developer Kernel
		\item [3. Ziffer] Patch-Level
		\item [Darauffolgende Ziffern] Bezeichnung, die im Makefile angegeben werden kann.
	\end{description}
\end{flashcard}

\begin{flashcard}[Command]{lsmod}
	Status der Module eines laufenden Kernels anzeigen.
	\begin{description}
		\item Greift auf \texttt{/proc/modules} zu
		\item [Example] \texttt{lsmod}
	\end{description}
\end{flashcard}

\begin{flashcard}[Command]{modinfo}
	Ein Modul des Kernels genauer unter die Lupe nehmen.
	\begin{description}
		\item  [Example] \texttt{modinfo storage}
		\item [Options]
			\item \texttt{-a} Autor 
			\item \texttt{-d} Beschreibung (description) 
			\item \texttt{-l} Lizenz
			\item \texttt{-p} Zu übergebende Parameter, falls möglich
			\item \texttt{-n} Namen der Datei
	\end{description}
\end{flashcard}
	
\begin{flashcard}[Command]{insmod}
	Ein Modul in den laufenden Kernel integrieren
	\begin{description}
		\item  [Example] \texttt{modinfo storage}
		\item [Options]
			\item \texttt{-a} Autor 
			\item \texttt{-d} Beschreibung (description) 
			\item \texttt{-l} Lizenz
			\item \texttt{-p} Zu übergebende Parameter, falls möglich
			\item \texttt{-n} Namen der Datei
	\end{description}
\end{flashcard}