\renewcommand{\sect}{101 Hardwareeinstellungen}

\cardfrontfoot{\chap/\sect}

\begin{flashcard}[Information]{Kernel-Version Identifizieren}
	Mehrere Möglichkeiten, wie z.B.:
	\begin{itemize}
		\item Command: \texttt{uname -r}. Output: Kernel-Version.
		\item \texttt{/lib/modules} \textrightarrow\ Kernel Historie.
		\item \texttt{/usr/src/<KERNEL-VERSION>} \textrightarrow\ Source Code.
	\end{itemize}
\end{flashcard}

\begin{flashcard}[Information]{Kernel-Version Codierung}
	\textbf{Beispiel}: 2.6.11.4-21.9-smp
	\begin{description}
		\item [1. Ziffer] Major release.
		\item [2. Ziffer] Minor release.
			\item Gerade Ziffern \textrightarrow\ Stable, Ungerade Ziffern \textrightarrow\ Developer Kernel.
		\item [3. Ziffer] Patch-Level.
		\item [Darauffolgende Ziffern] Bezeichnung, die im Makefile angegeben werden kann.
	\end{description}
\end{flashcard}

\begin{flashcard}[Command]{lsmod}
	Status der Module eines laufenden Kernels anzeigen.
	
	\begin{description}
		\item Greift auf \texttt{/proc/modules} zu.
		\item [Example] \texttt{lsmod}.
	\end{description}
\end{flashcard}

\begin{flashcard}[Command]{modinfo}
	Ein Modul des Kernels genauer unter die Lupe nehmen.
	
	\begin{description}
		\item  [Example] \texttt{modinfo storage}
		\item \textbf{Options}
		
		\begin{description}
			\item \texttt{-a} Autor 
			\item \texttt{-d} Beschreibung (description) 
			\item \texttt{-l} Lizenz
			\item \texttt{-p} Zu übergebende Parameter, falls möglich
			\item \texttt{-n} Namen der Datei
		\end{description}
	\end{description}
\end{flashcard}
	
\begin{flashcard}[Command]{insmod}
	Ein Modul in den laufenden Kernel integrieren.
	\begin{description}
		\item Benötigt Pfadangabe.
		\item Abhängigkeiten werden geprüft, aber nicht aufgelöst. \textrightarrow\ Fehlermeldung.
		\item Im Erfolgsfall keine Bestätigungsmeldung.
		\item  [Example] \texttt{insmod} \path{PATH_TO/usb-storage.ko}
		\item \textbf{Options}
		
		\begin{description}
			\item \todo{options here}
		\end{description}
	\end{description}
\end{flashcard}

\begin{flashcard}[Command]{rmmod}
	Ein Modul aus dem laufenden Kernel entfernen.
	\begin{description}
		\item Keine Pfadangabe nötig (arbeitet mit \path{/proc/modules}).
		\item Im Erfolgsfall keine Bestätigungsmeldung.
		\item Abhängigkeiten werden überprüft \textrightarrow\ Fehlermeldung wenn angegebenes Modul von anderem Modul benötigt wird.
		\item  [Example] \texttt{rmmod usb-storage}
		\item \textbf{Options}
		
		\begin{description}
			\item \texttt{-v} Verbose
			\item \texttt{-f} Entladen erzwingen (force), auch wenn Abhängigkeiten nicht erfüllt
		\end{description}
	\end{description}
\end{flashcard}

\begin{flashcard}[Command]{modprobe}
	Optimierte Kombination von insmod und rmmod.
	\begin{description}
		\item Kann Abhängigkeitsprobleme auch beheben.
		\item Keine Pfadangabe nötig. (Benutzt \texttt{uname -r}).
		\item Kann alle Module eines Typs laden.
		\item Im Erfolgsfall keine Bestätigungsmeldung. Nur im Fehlerfall.
		\item  [Example] \texttt{modprobe -at net}: 
		\item \textbf{Options}
		
		\begin{description}
			\item \texttt{-a} Alle Module
			\item \texttt{-t} Typ des Moduls
			\item \texttt{-r} Remove
			\item \texttt{-l} Alle ladbaren Module auflisten. (Deprecated).
			\item \texttt{-lt} Auflisten der Module eines Typs (z.B. \texttt{modprobe -lt fs})
		\end{description}
	\end{description}
\end{flashcard}

\begin{flashcard}[Command]{depmod}
	Informationen über Abhängigkeiten der Module sammeln.
	\begin{description}
		\item Abhängigkeiten der Module festgehalten in: \path{/lib/modules/KERNEL_VERSION/modules.dep}.
		\item [Ohne Optionen] \textrightarrow\ neue \texttt{modules.dep} erstellt.
		\item  [Example] \texttt{depmod} 
		\item \textbf{Options}
		
		\begin{description}
			\item \texttt{-n} Trockenlauf nach \texttt{stdout} (No new \texttt{modules.dep})
			\item \texttt{-A} Accelerated mode. Vor erstellung der Datei wird überprüft ob es Module gibt die neuer als \texttt{modes.dep} sind.
		\end{description}
	\end{description}
\end{flashcard}

\begin{flashcard}[File]{modules.dep}
	Beinhaltet Informationen über Abhängigkeiten der Module.
	\begin{description}
		\item Pfad: \path{/lib/modules/KERNEL_VERSION/modules.dep}.
		\item  [Format] \texttt{/PATH\_TO/Module1.ko:/PATH\_TO/Module2.ko}
		
		\begin{description}
			\item Module1 hängt von Module2 ab.
		\end{description}
	\end{description}
\end{flashcard}

\begin{flashcard}[Folder]{/proc/sys/kernel}
	Beinhaltet Informationen über Konfiguration des Kernels
	\begin{description}
		\item Wird zur Laufzeit angelegt. \textrightarrow\ Änderungen gehen bei reboot verloren.
		\item Befindet sich nicht auf der Festplatte, sondern bildet lediglich Informationen aus dem Arbeitsspeicher ab.
		\item \textbf{Examples}
		\begin{description}
			\item \texttt{cat /proc/sys/kernel/osrelease} \textrightarrow\ \texttt{3.13.0-24-generic}
			
			\item \texttt{cat /proc/sys/kernel/ostype} \textrightarrow\ \texttt{Linux}
			
			\item \texttt{cat /proc/sys/kernel/hostname} \textrightarrow\ \texttt{dwright}
			
			\item \texttt{cat /proc/sys/kernel/modprobe} \textrightarrow\ \texttt{/sbin/modprobe}
		\end{description}
	\end{description}
\end{flashcard}

\begin{flashcard}[Folder]{Kernel Source}
	Befindet sich unter \path{/usr/src/linux-KERNEL_VERSION}
	\begin{description}
		\item Softlink nach \path{/usr/src/linux} wird zudem erstellt.
	\end{description}
\end{flashcard}

\begin{flashcard}[Folder]{Statischer Teil des Lauffähigen Kernels}
	Befindet sich unter \path{/boot}
	\begin{description}
		\item Softlink nach \path{/boot/vmlinuz} wird zudem erstellt, der auf den tatsächlichen Namen des Kernels zeigt.
	\end{description}
\end{flashcard}

\begin{flashcard}[Folder]{Module des Kernels}
	Befindet sich unter \path{/lib/modules}
	\begin{description}
		\item Für jeden installierten Kernel ein Unterverzeichnis.
		\item Unterverzeichnisse nach der Kernel-Version benannt.
		\item \textbf{Examples}
		\begin{description}
			\item \path{/lib/modules/KERNEL_VERSION/fs} \textrightarrow\ Filesystem-Module
			
			\item \path{/lib/modules/KERNEL_VERSION/net} \textrightarrow\ Netzwerkkarten-Module
			
			\item \path{/lib/modules/KERNEL_VERSION/scsi} \textrightarrow\ SCSI-Adapter-Module
			
			\item \path{/lib/modules/KERNEL_VERSION/video} \textrightarrow\ Grafikkarten-Module
		\end{description}
	\end{description}
\end{flashcard}

\begin{flashcard}[TO DO]{\todo{AB SEITE 30!}}
\todo{AB SEITE 30!}
\end{flashcard}