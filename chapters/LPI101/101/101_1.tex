\renewcommand{\sect}{101 Hardwareeinstellungen}

\cardfrontfoot{\chap/\sect}

\begin{flashcard}[Information]{Kernel-Version Identifizieren}
	Mehrere Möglichkeiten, wie z.B.:
	\begin{itemize}
		\item Command: \texttt{uname -r}. Output: Kernel-Version.
		\item \texttt{/lib/modules} \textrightarrow\ Kernel Historie.
		\item \texttt{/usr/src/<KERNEL-VERSION>} \textrightarrow\ Source Code.
	\end{itemize}
\end{flashcard}

\begin{flashcard}[Information]{Kernel-Version Codierung}
	\textbf{Beispiel}: 2.6.11.4-21.9-smp
	\begin{description}
		\item [1. Ziffer] Major release.
		\item [2. Ziffer] Minor release.
			\item Gerade Ziffern \textrightarrow\ Stable, Ungerade Ziffern \textrightarrow\ Developer Kernel.
		\item [3. Ziffer] Patch-Level.
		\item [Darauffolgende Ziffern] Bezeichnung, die im Makefile angegeben werden kann.
	\end{description}
\end{flashcard}

\begin{flashcard}[Command]{lsmod}
	Status der Module eines laufenden Kernels anzeigen.
	
	\begin{description}
		\item Greift auf \texttt{/proc/modules} zu.
		\item [Example] \texttt{lsmod}.
	\end{description}
\end{flashcard}

\begin{flashcard}[Command]{modinfo}
	Ein Modul des Kernels genauer unter die Lupe nehmen.
	
	\begin{description}
		\item  [Example] \texttt{modinfo storage}
		\item \textbf{Options}
		
		\begin{description}
			\item \texttt{-a} Autor 
			\item \texttt{-d} Beschreibung (description) 
			\item \texttt{-l} Lizenz
			\item \texttt{-p} Zu übergebende Parameter, falls möglich
			\item \texttt{-n} Namen der Datei
		\end{description}
	\end{description}
\end{flashcard}
	
\begin{flashcard}[Command]{insmod}
	Ein Modul in den laufenden Kernel integrieren.
	\begin{description}
		\item Benötigt Pfadangabe.
		\item Abhängigkeiten werden geprüft, aber nicht aufgelöst. \textrightarrow\ Fehlermeldung.
		\item Im Erfolgsfall keine Bestätigungsmeldung.
		\item  [Example] \texttt{insmod} \path{PATH_TO/usb-storage.ko}
		\item \textbf{Options}
		
		\begin{description}
			\item \todo{options here}
		\end{description}
	\end{description}
\end{flashcard}

\begin{flashcard}[Command]{rmmod}
	Ein Modul aus dem laufenden Kernel entfernen.
	\begin{description}
		\item Keine Pfadangabe nötig (arbeitet mit \path{/proc/modules}).
		\item Im Erfolgsfall keine Bestätigungsmeldung.
		\item Abhängigkeiten werden überprüft \textrightarrow\ Fehlermeldung wenn angegebenes Modul von anderem Modul benötigt wird.
		\item  [Example] \texttt{rmmod usb-storage}
		\item \textbf{Options}
		
		\begin{description}
			\item \texttt{-v} Verbose
			\item \texttt{-f} Entladen erzwingen (force), auch wenn Abhängigkeiten nicht erfüllt
		\end{description}
	\end{description}
\end{flashcard}

\begin{flashcard}[Command]{modprobe}
	Optimierte Kombination von insmod und rmmod.
	\begin{description}
		\item Kann Abhängigkeitsprobleme auch beheben.
		\item Keine Pfadangabe nötig. (Benutzt \texttt{uname -r}).
		\item Kann alle Module eines Typs laden.
		\item Im Erfolgsfall keine Bestätigungsmeldung. Nur im Fehlerfall.
		\item  [Example] \texttt{modprobe -at net}: 
		\item \textbf{Options}
		
		\begin{description}
			\item \texttt{-a} Alle Module
			\item \texttt{-t} Typ des Moduls
			\item \texttt{-r} Remove
			\item \texttt{-l} Alle ladbaren Module auflisten. (Deprecated).
			\item \texttt{-lt} Auflisten der Module eines Typs (z.B. \texttt{modprobe -lt fs})
		\end{description}
	\end{description}
\end{flashcard}

\begin{flashcard}[Command]{depmod}
	Informationen über Abhängigkeiten der Module sammeln.
	\begin{description}
		\item Abhängigkeiten der Module festgehalten in: \path{/lib/modules/KERNEL_VERSION/modules.dep}.
		\item [Ohne Optionen] \textrightarrow\ neue \texttt{modules.dep} erstellt.
		\item  [Example] \texttt{depmod} 
		\item \textbf{Options}
		
		\begin{description}
			\item \texttt{-n} Trockenlauf nach \texttt{stdout} (No new \texttt{modules.dep})
			\item \texttt{-A} Accelerated mode. Vor erstellung der Datei wird überprüft ob es Module gibt die neuer als \texttt{modes.dep} sind.
		\end{description}
	\end{description}
\end{flashcard}

\begin{flashcard}[File]{modules.dep}
	Beinhaltet Informationen über Abhängigkeiten der Module.
	\begin{description}
		\item Pfad: \path{/lib/modules/KERNEL_VERSION/modules.dep}.
		\item  [Format] \texttt{/PATH\_TO/Module1.ko:/PATH\_TO/Module2.ko}
		
		\begin{description}
			\item Module1 hängt von Module2 ab.
		\end{description}
	\end{description}
\end{flashcard}

\begin{flashcard}[Folder]{/proc/sys/kernel}
	Beinhaltet Informationen über Konfiguration des Kernels
	\begin{description}
		\item Wird zur Laufzeit angelegt. \textrightarrow\ Änderungen gehen bei reboot verloren.
		\item Befindet sich nicht auf der Festplatte, sondern bildet lediglich Informationen aus dem Arbeitsspeicher ab.
	\end{description}
	
	\textbf{Examples}
	
	\begin{tabular}{ll}
		\path{cat /proc/sys/kernel/osrelease} 		& \texttt{4.4.0-57-generic}\\
		\path{cat /proc/sys/kernel/ostype} 		& \texttt{Linux}\\
		\path{cat /proc/sys/kernel/hostname} 	& \texttt{dwright-ubuntu}\\
		\path{cat /proc/sys/kernel/modprobe} 	& \texttt{/sbin/modprobe}\\
	\end{tabular}
\end{flashcard}

\begin{flashcard}[Folder]{Kernel Source}
	Befindet sich unter \path{/usr/src/linux-KERNEL_VERSION}
	\begin{description}
		\item Softlink nach \path{/usr/src/linux} wird zudem erstellt.
	\end{description}
\end{flashcard}

\begin{flashcard}[Folder]{Statischer Teil des Lauffähigen Kernels}
	Befindet sich unter \path{/boot}
	\begin{description}
		\item Softlink nach \path{/boot/vmlinuz} wird zudem erstellt, der auf den tatsächlichen Namen des Kernels zeigt.
	\end{description}
\end{flashcard}

\begin{flashcard}[Folder]{Module des Kernels}
	Befindet sich unter \path{/lib/modules}
	\begin{description}
		\item Für jeden installierten Kernel ein Unterverzeichnis.
		\item Unterverzeichnisse nach der Kernel-Version benannt.
		\item Vor Kernel-Version 2.6: Dateiendung \texttt{.o} (Object)
		\item Nach Kernel-Version 2.6: Dateiendung \texttt{.ko} (Kernel Object)
	\end{description}
	
	\textbf{Examples}
	
	\begin{tabular}{ll}
		\path{/lib/modules/KERNEL_VERSION/fs} 		& Filesystem-Module\\
		\path{/lib/modules/KERNEL_VERSION/net} 		& Netzwerkkarten-Module\\
		\path{/lib/modules/KERNEL_VERSION/scsi} 	& SCSI-Adapter-Module\\
		\path{/lib/modules/KERNEL_VERSION/video} 	& Grafikkarten-Module\\
	\end{tabular}
\end{flashcard}

\begin{flashcard}[Folder]{Wo und wie werden physikalische und logische Laufwerke dargestellt?}
	Als Gerätedateien unterhalb von \texttt{/dev}.
	
	\begin{tabular}{ll}
		\textbf{(E)IDE Laufwerke}	& \texttt{/dev/hdX}\\
		\textbf{SCSI Laufwerke}		& \texttt{/dev/sdX}\\
		\textbf{SATA Laufwerke}		& \texttt{/dev/sdX}\\
	\end{tabular}
	
	\caution{Manche Debian Ableger (Ubuntu, Mint, ...) sprechen auch IDE Festplatten unter \texttt{/dev/sdX} an!}
\end{flashcard}

\begin{flashcard}[Information]{Wie viele Kanäle haben die meisten (E)IDE Controller?\\Wie viele Geräte können daran angeschlossen werden?\\Wie heissen die Gerätedateien?}
	Zwei Kanäle mit je einem Master und Slave-Laufwerk.
	
	\begin{tabular}{ll}
		\textbf{Primary Master} 	& \texttt{/dev/hda}\\
		\textbf{Primary Slave} 		& \texttt{/dev/hdb}\\
		\textbf{Secondary Master} 	& \texttt{/dev/hdc}\\
		\textbf{Secondary Slave} 	& \texttt{/dev/hdd}\\
	\end{tabular}
\end{flashcard}

\begin{flashcard}[Information]{Wie viele Partitionen können auf einer Festplatte erstellt werden?}
	Es können gleichzeitig bis zu 4 primäre Partitionen auf einer Festplatte existieren. Erweiterte Partitionen können nicht direkt angesprochen werden. Sie sind nur ein Behälter für Logische Partitionen.
	\begin{tabular}{lll}
		& \textbf{IDE} & \textbf{SCSI/SATA}\\
		\textbf{Primäre Partitionen}	& 4 	& 4 \\
		\textbf{Erweiterte Partitionen} & 1		& 1 \\
		\textbf{Logische Partitionen} & 60 & 12\\
	\end{tabular}
\end{flashcard}

\begin{flashcard}[Information]{Wie funktioniert die Nummerierung von Partitionen auf einer Festplatte?}
	
	\begin{description}
	\item Primäre und erweiterte Partitionen werden von 1 bis 4 durchnummeriert. 
	\item Die erste logische Partition bekommt \textbf{IMMER} die Nummer 5.
	\end{description}

	
	\textbf{Example}
	
	\begin{tabular}{ll}
		\texttt{/dev/hda1} & 1. primäre Partition\\
		\texttt{/dev/hda2} & 2. primäre Partition\\
		\texttt{/dev/hda3} & einzige erweiterte Partition\\
		\texttt{/dev/hda5} & 1. logische Partition\\
		\texttt{/dev/hda6} & 2. logische Partition\\
		\texttt{/dev/hda7} & 3. logische Partition\\
	\end{tabular}
\end{flashcard}

\begin{flashcard}[Folder]{In welchem Verzeichnis wird die aktuelle Konfiguration der Hardwareressourcen abgebildet?}
	\path{/proc}
\end{flashcard}

\begin{flashcard}[Folder]{Wo befinden sich Informationen über die vom System verwendeten Interrupts?}
	\path{/proc/interrupts}
\end{flashcard}

\begin{flashcard}[Folder]{Wo befinden sich Informationen über die vom System verwendeten I/O-Adressen?}
	\path{/proc/ioports}
	
	\lstinputlisting[style=bash, title=Example]{./src/procIOPortsInfo.sh}
\end{flashcard}

\begin{flashcard}[Folder]{Wo befinden sich Informationen über die von Geräten verwendeten DMA-Kanäle?}
	\path{/proc/dma}
\end{flashcard}

\begin{flashcard}[Folder]{Wo befinden sich Informationen über die den PCI-Bus?}
	
	\begin{tabular}{ll}
		\texttt{/proc/pci} & Bei älteren Kernel-Versionen\\
		\texttt{/proc/bus/pci} & Bei neueren Kernel-Versionen\\
	\end{tabular}
\end{flashcard}

\begin{flashcard}[Folder]{Wo befinden sich Informationen über angeschlossene SCSI-Geräte?}
	\path{/proc/scsi}
	
	\lstinputlisting[style=bash, title=Example]{./src/procScsiInfo.sh}
\end{flashcard}

\begin{flashcard}[Folder]{Wo befinden sich Informationen über angeschlossene USB-Geräte?}
	\path{/proc/scsi/usb-storage}
	
	USB-Geräte werden ähnlich gehandhabt wie SCSI-Geräte und deshalb auch in diesem Ordner abgebildet.
	
	\lstinputlisting[style=bash, title=Example]{./src/procUSBInfo.sh}
\end{flashcard}

\begin{flashcard}[Command]{Wie findet man informationen über den PCI-Bus?}
	\texttt{lspci}

	\begin{description}	
	\item \textbf{Options}
	
		\begin{description}
			\item \texttt{-v} Verbose (3 Stufen möglich)
			
			\item \texttt{-t} Tree, stellt die informationen in Baumstruktur dar
		\end{description}
	\end{description}
	\lstinputlisting[style=bash, title=Example]{./src/lspci.sh}
\end{flashcard}

\begin{flashcard}[Information]{Wie heissen die USB-Controller Typen und deren Treiber?}
	\begin{tabular}{lll}
		USB 1.1 & OHCI, UHCI & \texttt{usb-ohci.o}, \texttt{usb-uhci.o}\\
		USB2.0 & EHCI & \texttt{usb-ehci.o}\\
	\end{tabular}
	
	Die Abkürzungen stehen für: \{Open, Universal, Enhanced\} Host Controller Interface
\end{flashcard}


\begin{flashcard}[Information]{Nenne ein paar Beispiele für USB-Klassen und deren Klassenreiber?}
	\begin{tabular}{ll}
		HID 		& \texttt{usbhid.ko}\\
		Storage 	& \texttt{usb-storage.ko}\\
		Bluetooth 	& \texttt{btusb.ko}
	\end{tabular}
	
	Die Abkürzungen stehen für: \{Open, Universal, Enhanced\} Host Controller Interface
\end{flashcard}

\begin{flashcard}[Command]{Wie findet man Informationen über geladene Module und deren Treiber?}
	\texttt{lsmod}
	
	\begin{description}
	\item Die Spalte ganz rechts listet die Treiber auf.
	
	\item Filtern mit z.B. \texttt{lsmod | grep usb}
	\end{description}
	
	\lstinputlisting[style=bash, title=Example]{./src/lsmod.sh}
	
	
\end{flashcard}

\begin{flashcard}[Command]{Wie findet man Informationen über angeschlossene USB-Geräte?}
	\texttt{lsusb}
	
	\begin{description}
		\item \textbf{Options}
		
		\begin{description}
			\item \texttt{-v} Verbose
			
			\item \texttt{-t} Tree, stellt die informationen in Baumstruktur dar
			
			\item \texttt{-s [[bus]:][dev]} Nur spezifizierte Bus und/oder Device Nr. anzeigen.
			
			\item \texttt{-d [ven]:[prod]} Nur spezifizierte Vendor und Product ID anzeigen.
		\end{description}
	\end{description}
	
	\lstinputlisting[style=bash, title=Example]{./src/lsusb.sh}
\end{flashcard}

\begin{flashcard}[Daemon]{Welche Mechanismen gibt es, um USB-Treiber automatisch zu laden?\\Welcher ist weiter verbreitet?}	
	\begin{tabular}{ll}
	\texttt{usbmgr} 	& \\
	\texttt{hotplug} 	& \textleftarrow\ Wesentlich weiter verbreitet\\
	\end{tabular}
\end{flashcard}


\begin{flashcard}[TO DO]{\todo{AB SEITE 36}}	
	\todo{AB SEITE 36}
\end{flashcard}