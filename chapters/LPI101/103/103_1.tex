\renewcommand{\sect}{103 GNU- und Unix-Kommandos}

\cardfrontfoot{\chap/\sect}

\begin{flashcard}[Information]{Was sind die Komponenten eines Kommandos?}
	Ein Kommando hat bis zu 3 Komponenten
	
	\begin{tabular}{ll}
		\textbf{Komponente}	& \textbf{Beispiel}\\
		Das Kommando selbst & \texttt{mount}\\
		Optionen 			& \texttt{mount -a}\\
		Argumente			& \texttt{mount /dev/hda1 /boot}
	\end{tabular}
	
	\caution{In den Beispielen werden Argumente und Optionen im Kontext eines Kommandos dargestellt. Mount gehört nicht zur Option oder zum Argument!}
\end{flashcard}

\begin{flashcard}[Information]{Was sind gängige Methoden zur Übergabe von Optionen an Kommandos?}
	\begin{description}
		\item Mit vorangehendem Bindestrich
		
		\item Ohne vorangehenden Bindestrich
		
		\item Ganze Wörter (in der Regel mit zwei vorangehenden Bindestrichen)
	\end{description}
\end{flashcard}

\begin{flashcard}[Information]{Was sind Unterschiede zwischen Shellvariabeln und Umgebungsvariabeln?}
	\begin{tabular}{ll}
		\textbf{Umgebungsvariabeln} & \textbf{Shellvariabeln}\\
		Gelten für alle Shells  & Müssen für jede Subshell neu deklariert werden\\
		Üblicherweise Grossbuchstaben	& Üblicherweise Kleinbuchstaben
	\end{tabular}
\end{flashcard}

\begin{flashcard}[File]{Was ist die Aufgabe von \path{/etc/profile}?}
	\begin{description}
		\item Konfigurationsdatei für Umgebungsvariabeln und erste \texttt{PATH}-Anweisung.
		
		\item Wird bei der Anmeldung eines Benutzers gelesen \textrightarrow Bei Änderungen Neuanmeldung nötig
	\end{description}
\end{flashcard}

\begin{flashcard}[File]{Was ist die Aufgabe von \path{/etc/.bashrc}?}
	\begin{description}
		\item Konfigurationsdatei für systemweite Einstellungen, Umgebungsvariabeln Aliases und Funktionen.
		
		\item Wird beim Start jeder neuen Shell gelesen.
	\end{description}
\end{flashcard}

\begin{flashcard}[File]{Was ist die Aufgabe von \path{~/.bash_profile}?}
	\begin{description}
		\item Nicht immer vorhanden.
		
		\item Konfigurationsdatei für benutzerspezifische Umgebungsvariabeln, weitere \texttt{PATH}-Anweisungen und den zu verwendeten Standard-Editor.
		                                                                                                      
		\item Wird bei der Anmeldung eines Benutzers sofort nach \path{/etc/profile} gelesen \textrightarrow Bei Änderungen Neuanmeldung nötig 
	\end{description}
\end{flashcard}

\begin{flashcard}[File]{Was ist die Aufgabe von \path{~/.bash_login}?}
	\begin{description}
		\item Alternative zu \path{~/.bash_profile}. Wird nur gelesen wenn diese nicht vorhanden ist. Inhalt und Zweck sind in beiden Files gleich.
		
		\item Nicht immer vorhanden.
		
		\item Konfigurationsdatei für benutzerspezifische Umgebungsvariabeln, weitere \texttt{PATH}-Anweisungen und den zu verwendeten Standard-Editor.
		
		\item Wird bei der Anmeldung eines Benutzers sofort nach \path{/etc/profile} gelesen \textrightarrow Bei Änderungen Neuanmeldung nötig 
	\end{description}
\end{flashcard}

\begin{flashcard}[File]{Was ist die Aufgabe von \path{~/.profile}?}
	\begin{description}
		\item Ursprüngliche Konfigurationsdatei der Bash.
		
		\item Wird nur gelesen wenn weder \path{~/.bash_profile} noch \path{~/.bash_login} vorhanden sind. Inhalt und Zweck sind in allen drei Files gleich.
		
		\item Nicht immer vorhanden.
		
		\item Konfigurationsdatei für benutzerspezifische Umgebungsvariabeln, weitere \texttt{PATH}-Anweisungen und den zu verwendeten Standard-Editor.
		
		\item Wird bei der Anmeldung eines Benutzers sofort nach \path{/etc/profile} gelesen \textrightarrow Bei Änderungen Neuanmeldung nötig 
	\end{description}
\end{flashcard}

\begin{flashcard}[File]{Was ist die Aufgabe von \path{~/.bashrc}?}
	\begin{description}
		\item Ursprüngliche Konfigurationsdatei der Bash.
		
		\item Wird beim Start jeder neuen Shell gelesen.
		
		\item Konfigurationsdatei für Aliases und Funktionen.
	\end{description}
\end{flashcard}

\begin{flashcard}[File]{Was ist die Aufgabe von \path{~/.bash_logout}?}
	\begin{description}
		\item Optionale Datei, die bei der Abmeldung eines Benutzers gelesen werden.
		
		\item Z.B. könnte sie den Monitor löschen.
	\end{description}
\end{flashcard}

\begin{flashcard}[Command]{Wie können die gesetzten Shellvariabeln angezeigt werden?}
	\texttt{set}
	
	\begin{description}
		\item \textbf{Options}
		
		\begin{description}
			\item \todo{options here}
		\end{description}
	\end{description}
	
	\lstinputlisting[style=bash, title=Example]{./src/set.sh}
\end{flashcard}

\begin{flashcard}[Command]{Wie kann in der Kommandozeile eine Variable deklariert werden?}
	\lstinputlisting[style=bash, title=Example]{./src/shellVar.sh}
	
	Wie im Beispiel sieht wird die Variable nicht automatisch an Subshells vererbt. Um die Variable zu vererben, muss vor dem aufruf der Bash das Kommando \texttt{export x} ausgeführt werden.
\end{flashcard}

\begin{flashcard}[Command]{Wie kann eine Variable exportiert werden?}
	\lstinputlisting[style=bash, title=Example]{./src/export.sh}
	\caution{Prüfungsrelevant: Eine Variable kann niemals in eine übergeordnete Shell exportiert werden! Nur in eine Subshell.}
\end{flashcard}

\begin{flashcard}[Command]{Wie kann der Inhalt einer Variabeln zurückgesetzt werden?}
	\lstinputlisting[style=bash, title=Example]{./src/unset.sh}
\end{flashcard}

\begin{flashcard}[Variable]{Was ist die Aufgabe von \texttt{\$HISTSIZE} und wo wird sie definiert?}
	\begin{description}
		\item Beinhaltet die Anzahl Kommandos, die in der Befehls-History gespeichert werden.
		
		\item Wird normalerweise in \path{/etc/profile} festgelegt.
	\end{description}
\end{flashcard}

\begin{flashcard}[Variable]{Was ist die Aufgabe von \texttt{\$PS1} und wo wird sie definiert?}
	\begin{description}
		\item Legt das aussehen des Command-Prompts fest
		
		\item Wird normalerweise in \path{/etc/bashrc} festgelegt.
	\end{description}
	
	\lstinputlisting[style=bash, title=Example]{./src/ps1.sh}
	
	Note: \textbackslash u steht für User, @ wird so ausgegeben, \textbackslash h steht für Host, Doppelpunkt wird auch so ausgegeben, \textbackslash w steht für Working Directory, \\\$ gibt das Dollar-Zeichen aus.
\end{flashcard}

\begin{flashcard}[Variable]{Was ist die Aufgabe von \texttt{\$?}, \texttt{\$1}, \texttt{\$2}, usw?}
	\begin{description}
		\item \texttt{\$?} beinhaltet den Error-Level des zuletzt ausgeführten Commands. z.B. 0 für kein Fehler, 127 für Command not found
		
		\item \texttt{\$1}, \texttt{\$2}, usw beinhalten die 1., 2., ... Übergabeparameter eines Programms.
	\end{description}
\end{flashcard}

\begin{flashcard}[Command]{Welche Methoden gibt es, um die Command-History zu verwenden?}
	\begin{tabular}{lp{0.6\linewidth}}
		\texttt{cursor up}			& Kommandos werden in umgekehrter Reihenfolge angezeigt.\\
		
		\texttt{!!}					& Bang-Bang, führt letzten Befehl der History nochmals aus.\\
		
		\texttt{history}			& Gibt nummerierte Liste mit den zuletzt verwendeten Befehlen aus.\\
		
		\texttt{!n-}				& Führt das n-te Kommando aus der nummerierten Liste von \texttt{history} aus. (Beginnt oben) \\

		\texttt{!-n}				& Führt das letzte Kommando - n aus. \texttt{!-2} führt das vorletzte Kommando aus.\\
				
		\texttt{!<STRING>}	& führt das letzte Kommando aus, das mit \texttt{<STRING>} beginnt.\\
		
		\texttt{!?<STRING>}	& Führt das letzte Kommando aus, in dem \texttt{<STRING>} vorkommt.
	\end{tabular}
\end{flashcard}

\begin{flashcard}[Command]{Was ist der Ablauf, wenn ein Kommando ohne angabe des Pfades ausgeführt wird?}
	\begin{enumerate}
		\item Shell prüft, ob es ein internes Kommando der Shell ist, wie z.B. \texttt{echo}, \texttt{bg}, \texttt{fg} etc.
		
		\item Falls kein passendes Kommando gefunden wird, wird die \texttt{PATH}-Variable durchsucht.
		
		\item Falls mehrere mit dem selben Namen gefunden werden, wird das erste ausgeführt. In einem solchen Fall kann das Kommando \texttt{which} verwendet werden.
	\end{enumerate}
	
	\caution{Ein Programm wird auch dann nicht ohne Pfadangabe ausgeführt, wenn man sich aktuell in dessen Verzeichnis befindet. \textrightarrow \texttt{./} vorangeben.}
	
	\lstinputlisting[style=bash, title=Example]{./src/which.sh}
\end{flashcard}


\begin{flashcard}[Command]{Wie kann man das aktuelle Verzeichnis ausgeben?}
	\texttt{pwd}
	
	\lstinputlisting[style=bash, title=Example]{./src/pwd.sh}
\end{flashcard}

\begin{flashcard}[Command]{Wie kann man den Inhalt der \texttt{PATH}-Variable ausgeben?}
	\texttt{echo \$PATH}
	
	\lstinputlisting[style=bash, title=Example]{./src/path.sh}
\end{flashcard}

\begin{flashcard}[Command]{Wie benutzt man die Manpages?}
	\texttt{man <COMMAND>}
	
	\lstinputlisting[style=bash, title=Example]{./src/man.sh}
\end{flashcard}

\begin{flashcard}[Command]{Mit welchem Kommando findet man den aktuellen \texttt{MANPATH} und das Cache-Verzeichnis von \texttt{man}?}
	\texttt{manpath}
	
	\begin{description}
		\item Für \texttt{MANPATH}: Ohne Optionen
		
		\item Für Cache-Verzeichnis (\texttt{catpath}):  \texttt{manpath -c}
	\end{description}
\end{flashcard}

\begin{flashcard}[Information]{Was sind die Sektionen der Manpages?}
	\begin{tabular}{lp{0.6\linewidth}}
		1 & Ausführbare Programme für den Benutzer \\
		2 & Systemaufrufe (Funktionen, die durch den Kernel unterstützt werden)\\
		3 & Bibliothekaufrufe (Libraries)\\
		4 & besondere Dateien (normalerweise in \texttt{/dev})\\
		5 & Dateiformate und Konventionen\\
		6 & Spiele\\
		7 & Sonstiges (Makro-Pakete und Konventionen)\\
		8 & administrative Programme (nur für Root)\\
		9 & Kernel-Routinen\\
		A & mehrteilige Manpages
	\end{tabular}
	
	Falls ein Programm in mehrere dieser Sektionen fällt ist der Aufruf für das Kommando: \texttt{man <SECTION> <COMMAND>}.
\end{flashcard}

\begin{flashcard}[Information]{Wie ist der Aufbau der Manpages?}
	\begin{tabular}{lp{0.6\linewidth}}
		\textit{Name} & Bezeichnung des Elementes und kurze Beschreibung\\
		\textit{Synopsis} & Vollständige Kurzbeschreibung der Syntax\\
		\textit{Description} & Ausführliche Beschreibung des Elementes\\
		\textit{Overview} & Überblick über die Komplexeren Zusammenhänge\\
		\textit{Options} & Optionen und deren Beschreibung\\
		\textit{Return Values} & Rückgabewerte aka Exit-Status\\
		\textit{See also} & Verweise auf artverwandte Themen\\
		\textit{Bugs} & Bekannte Fehler\\
		\textit{Files} & Konfigurationsdateien u. Ä.\\
	\end{tabular}
\end{flashcard}

\begin{flashcard}[Command]{Wie kann man die Manpages nach Programmen durchsuchen?}
	\begin{tabular}{lp{0.5\linewidth}}
		\texttt{whatis <COMMAND>} & Durchsucht nur das Feld \textit{Name}.\\
		\texttt{apropos <COMMAND\_OR\_DESCRIPTION>} & Durchsucht die Felder \textit{Name} und \textit{Description}.
	\end{tabular}
	
	\lstinputlisting[style=bash, title=Example]{./src/whatis_apropos.sh}
\end{flashcard}


\begin{flashcard}[Command]{Wie findet man Informationen über die Verzeichnisse eines Programms, dessen Konfigurationsdateien und Manpages?}
	\texttt{whereis}
	
	\lstinputlisting[style=bash, title=Example]{./src/whereis.sh}
\end{flashcard}

\begin{flashcard}[Command]{Wie können Textdateien in der Shell ausgegeben, an ein Programm umgeleitet oder miteinander verbunden werden?}
	\texttt{cat}
	
	\lstinputlisting[style=bash, title=Example]{./src/cat.sh}
\end{flashcard}

\begin{flashcard}[Command]{Wie können Textdateien umgekehrt ausgegeben werden?}
	\texttt{tac}
\end{flashcard}

\begin{flashcard}[Command]{Wie kann man die ersten \texttt{n} Zeilen einer Textdatei anzeigen?}
	\texttt{head -n  <FILE>}
	
	Ohne Option werden per Default die ersten 10 Zeilen ausgegeben.
	
	Es können auch mehrere Files angegeben werden.
	
	\lstinputlisting[style=bash, title=Example]{./src/head.sh}
\end{flashcard}

\begin{flashcard}[Command]{Wie kann man die letzten \texttt{n} Zeilen einer Textdatei anzeigen?}
	\texttt{head -n  <FILE>}
	
	Ohne Option werden per Default die letzten 10 Zeilen ausgegeben.
	
	Es können auch mehrere Files angegeben werden.
	
	\lstinputlisting[style=bash, title=Example]{./src/tail.sh}
\end{flashcard}

\begin{flashcard}[Command]{Wie kann man die letzten Zeilen einer Textdatei fortlaufend anzeigen während ein anderes Programm auf diese Datei schreibt?}
	\texttt{tail -f <FILE>}
	
	Besonders beliebt (auch für die Prüfung) mit \texttt{/var/log/syslog} oder \texttt{/var/log/messages}.
\end{flashcard}

\begin{flashcard}[Command]{Wie kann man den Befehl \texttt{less} verwenden?}
	\texttt{less <FILE>}
	
	\texttt{less} ist ein Pager, der ein Textfile anzeigen kann. Es liest nicht immer das ganze File ein, und kann nach oben und unten navigieren, im Gegensatz zum älteren \texttt{more}.
	
	\textbf{Befehlsbeispiele:}
	
	\begin{description}
		\item \texttt{/<STRING>} Vorwärtssuche, sprint zum nächsten Auftreten von \texttt{STRING} in Vorwärtsrichtung.
		
		\item \texttt{?<STRING>} Rückwärtssuche, sprint zum nächsten Auftreten von \texttt{STRING} in Rückwärtsrichtung.
		
		\item \texttt{n} Springt zum nächsten Auftreten der letzten Suche (abhängig von Suchrichtung / oder ?).
		
		\item \texttt{N} Springt zum vorhergehenden Auftreten der letzten Suche (abhängig von Suchrichtung / oder ?).
	\end{description}
\end{flashcard}

\begin{flashcard}[Command]{Wie können in einem Textfile Tabstopps in Leerzeichen umgewandelt werden oder umgekehrt?}
	\begin{tabular}{cc}
		tabs \textrightarrow spaces & spaces \textrightarrow tabs\\
		\texttt{expand} & \texttt{unexpand}
	\end{tabular}
	
	\vspace{0.5cm}
	
	Für beide Kommandos ist standardmässig das Äquivalent für einen Tabstopp 8 Leerzeichen. Dies kann mit der Option -t angepasst werden.
	
	\vspace{0.5cm}
	
	Hier bietet es sich an, die Ausgabe von \texttt{stdout} in ein File umzuleiten:
	
	\texttt{expand textfile1 > textfile2}.
\end{flashcard}

\begin{flashcard}[Command]{fmt}
	\begin{description}
		\item Textformatierungsprogramm.
		
		\item Entfernt automatisch Zeilenumbrüche.
		
		\item Parameter -w um die Breite (width) des Textes anzugeben.
	\end{description}
\end{flashcard}

\begin{flashcard}[Command]{nl}
	\begin{description}
		\item Nummeriert die Zeilen eines Dokumentes (Number Lines).
		
		\item Ausgabe Standardmässig nach \texttt{stdout}, Umleitung in anderes Dokument von Vorteil.
		
		\item \caution{Nicht zu verwechseln mit \texttt{ln} (Links erstellen).}
	\end{description}
\end{flashcard}

\begin{flashcard}[Command]{Wie kann man eine Datei für den Druck vorbereiten?}
	\texttt{pr}
	
	\textbf{Options:}
	
	\begin{tabular}{lp{0.8\linewidth}}
		\texttt{-h <HEADER\_STRING>}	& Titel für Druckheaderzeile \\
		\texttt{-W} 					& Seitenbreite (width)\\
		\texttt{-l}						& Seitenlänge\\
	\end{tabular}
	
	\lstinputlisting[style=bash, title=Example]{./src/pr.sh}
\end{flashcard}

\begin{flashcard}[Command]{wc}
	Zählt die Anzahl Wörter einer oder mehreren Dateien.
	
	\textbf{Options:}
	
	\begin{tabular}{lp{0.8\linewidth}}
		\texttt{-c} & Zeigt nur die Anzahl der Bytes.\\
		\texttt{-l} & Zeigt nur die Anzahl der Zeilen (lines).\\
		\texttt{-w} & Zeigt nur die Anzahl der Wörter.\\
		\texttt{-m} & Zeigt nur die Anzahl der Zeichen.\\
	\end{tabular}
\end{flashcard}

\begin{flashcard}[Command]{hexdump}
	Dateien in Hexadezimal-, Dezimal-, Oktal-, oder ASCII-Format anzeigen.
	
	\textbf{Options:}
	
	\begin{tabular}{lp{0.8\linewidth}}
		\texttt{-b} & One-byte octal display. Stellt jeweils ein Byte im Oktalformat dar.\\
		\texttt{-c} & One-byte character display. Stellt jeweils ein Byte als Charakter dar.\\
		\texttt{-C} & Canonical hex+ASCII display. Stellt jeweils ein Byte im Hex-Format dar, und fügt am Schluss die Zeile in ASCII-Format hinzu.\\
		\texttt{-d} & Two-byte decimal display. Stellt jeweils zwei Bytes im Dezimalformat dar.\\
	\end{tabular}
	
	\lstinputlisting[style=bash, title=Example]{./src/hexdump.sh}
\end{flashcard}

\begin{flashcard}[Command]{od}
	Octaldump. Dateien in Hexadezimal-, Dezimal-, Oktal-, oder ASCII-Format anzeigen.
	
	\textbf{Options:}
	
	\begin{tabular}{lp{0.8\linewidth}}
		\texttt{-x} & Two-byte hex display. Stellt jeweils zwei Bytes im Hex-Format dar.\\
		\texttt{-a} & One-byte character display. Stellt jeweils ein Byte als Charakter dar.\\
		\texttt{-b} & One-byte octal display. Stellt jeweils ein Byte im Oktal-Format dar.\\
		\texttt{-d} & Two-byte decimal display. Stellt jeweils zwei Bytes im Dezimalformat dar.\\
	\end{tabular}
\end{flashcard}

\begin{flashcard}[Command]{sort}
	Zeilen in einer Datei oder andere Eingabequellen sortieren.
	
	\textbf{Options:}
	
	\begin{tabular}{lp{0.8\linewidth}}
		\texttt{-n} & Sortiert nach numerischen Kriterien\\
		\texttt{-o} & Outfile. Ausgabe in eine Datei. Standardmässig wird nach \texttt{stdout} ausgegeben.\\
		\texttt{-r} & Reverse. Ausgabe in umgekehrter Reihenfolge.\\
	\end{tabular}
\end{flashcard}

\begin{flashcard}[Command]{uniq}
	Wiederholte Zeilen in einer Datei löschen. Voraussetzung ist, dass diese Zeilen auf einander folgen.
	
	Sollten die Zeilen nicht auf einander folgen, kann z.B. \texttt{sort} vorgeschalten werden.
\end{flashcard}

\begin{flashcard}[Command]{Wie können Dateien in kleinere Dateien aufgeteilt, und wieder zusammengefügt werden?}
	\texttt{split} und \texttt{cat}
	
	\begin{description}
		\item	\texttt{split} arbeitet per Default in zeilenorientiert (ohne Optionen aufgeteilt in Dateien mit je 1000 Zeilen). 
		
		\item	Mit der Option \texttt{-l} kann die Anzahl Zeilen (lines) und mit \texttt{-b} die Anzahl Bytes für die aufgeteilten Dateien festgelegt werden
	\end{description}
	
	\lstinputlisting[style=bash, title=Example]{./src/split_cat.sh}
\end{flashcard}

\begin{flashcard}[Command]{cut}
	Spalten aus einer Textdatei ausschneiden.
	
	\textbf{Options:}
	
	\begin{tabular}{lp{0.8\linewidth}}
		\texttt{-d} & Delimiter.\\
		\texttt{-f} & Fields. Definiert welche Spalten ausgeschnitten werden sollen.\\
	\end{tabular}
\end{flashcard}

\begin{flashcard}[Command]{paste}
	Textdateien nebeneinander zusammenfügen
\end{flashcard}

\begin{flashcard}[Command]{join}
	Textdateien nebeneinander zusammenfügen, kann im Gegensatz zu \texttt{paste} einen gemeinsamen Nenner der verknüpften Dateien Festlegen.
	
	\textbf{Options:}
	
	\begin{tabular}{lp{0.8\linewidth}}
		\texttt{-t} & Delimiter.\\
		\texttt{-j} & Definiert den gemeinsamen Nenner.
	\end{tabular}
\end{flashcard}

\begin{flashcard}[Command]{tr}
	Einzelne Zeichen durch andere ersetzen oder löschen.
	
	Es können keine Textdateien als Argument übergeben werden, deshalb müssen solche mit \texttt{cat foo.txt | tr x y} übergeben werden.
	
	\textbf{Options:}
	
	\begin{tabular}{llp{0.8\linewidth}}
		\texttt{-d} & \texttt{--delete} & Löschen des Zeichens.\\
		\texttt{-c} & \texttt{--complement} & Kehrt die Ausgabe ins Gegenteil.\\
		\texttt{-s} & \texttt{--squeeze-repeats} & Unterdrückt wiederholende Zeichen.
	\end{tabular}
	
	\lstinputlisting[style=bash, title=Example]{./src/tr.sh}
\end{flashcard}

\begin{flashcard}[Command]{Wie kann der Inhalt eines Verzeichnisses aufgelistet werden?}
	\texttt{ls}
	
	\textbf{Options:}
	
	\begin{tabular}{lp{0.8\linewidth}}
		\texttt{-l} & Listing Format, beinhaltet u.A. Zugriffsrechte und Timestamps.\\
		\texttt{-i} & Zeigt die verwendeten Inodes.\\
		\texttt{-a} & Zeigt alle Dateien an, auch hidden Files/Folders.\\
		\texttt{-s} & Zeigt die Grösse jeder Datei in Blocks an. Setzt \texttt{-l} voraus.\\
		\texttt{-h} & Zeigt die Grösse jeder Datei in Menschenlesbarem Format an (z.B. 21KB, 24MB, 3GB). Setzt \texttt{-l} voraus.
	\end{tabular}
	
	\lstinputlisting[style=bash, title=Example]{./src/ls.sh}
\end{flashcard}

\begin{flashcard}[Command]{Wie kann das aktuelle Verzeichnis gewechselt werden?}
	\texttt{cd}	(Change Directory)
	
	Bei verwendung des Kommandos ohne Optionen und Argumente wechselt \texttt{cd} in das Home Directory
	
	\lstinputlisting[style=bash, title=Example]{./src/cd.sh}
\end{flashcard}

\begin{flashcard}[Command]{Wofür steht die Tilde (\texttildelow)?}
	Für das Heimatverzeichnis (home directory) des aktuellen Users.
\end{flashcard}

\begin{flashcard}[Command]{Wie können Dateien und Verzeichnisse kopiert werden?}
	\texttt{cp}
	
	Es müssen immer Quell- und Zielobjekte angegeben werden. Dabei ist das letzte immer das Ziel.
	
	\textbf{Options:}
	
	\begin{tabular}{llp{0.65\linewidth}}
		\texttt{-i} & \texttt{--interactive} & Fragt vor dem Überschreiben bei eventuell bereits existierender Zielobjekte nach..\\
		\texttt{-f} & \texttt{--force} & Erzwingt den Schreibvorgang im Zielverzeichnis.\\
		\texttt{-p} & \texttt{--preserve} & Kopiert Dateien unter Beibehaltung des Eigentümers, der Eigentümergruppe, der Berechtigungen und Timestamps.\\
		\texttt{-R} \texttt{-r} & \texttt{--recursive} & Kopiert ein Verzeichnis inklusive aller Unterverzeichnisse und Dateien.
	\end{tabular}
\end{flashcard}

\begin{flashcard}[TO DO]{\todo{AB SEITE 115 (mv)}}	
	\todo{AB SEITE 115 (mv)}
\end{flashcard}