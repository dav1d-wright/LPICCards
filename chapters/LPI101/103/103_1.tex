\renewcommand{\sect}{103 GNU- und Unix-Kommandos}

\cardfrontfoot{\chap/\sect}

\begin{flashcard}[Information]{Was sind die Komponenten eines Kommandos?}
	Ein Kommando hat bis zu 3 Komponenten
	
	\begin{tabular}{ll}
		\textbf{Komponente}	& \textbf{Beispiel}\\
		Das Kommando selbst & \texttt{mount}\\
		Optionen 			& \texttt{mount -a}\\
		Argumente			& \texttt{mount /dev/hda1 /boot}
	\end{tabular}
	
	\caution{In den Beispielen werden Argumente und Optionen imm Kontext eines Kommandos dargestellt. Mount gehört nicht zur Option oder zum Argument!}
\end{flashcard}

\begin{flashcard}[Information]{Was sind gängige Methoden zur Übergabe von Optionen an Kommandos?}
	\begin{description}
		\item Mit vorangehendem Bindestrich
		
		\item Ohne vorangehenden Bindestrich
		
		\item Ganze Wörter (in der Regel mit zwei vorangehenden Bindestrichen)
	\end{description}
\end{flashcard}

\begin{flashcard}[Information]{Was sind Unterschiede zwischen Shellvariabeln und Umgebungsvariabeln?}
	\begin{tabular}{ll}
		\textbf{Umgebungsvariabeln} & \textbf{Shellvariabeln}\\
		Gelten für alle Shells  & Müssen für jede Subshell neu deklariert werden\\
		Üblicherweise Grossbuchstaben	& Üblicherweise Kleinbuchstaben
	\end{tabular}
\end{flashcard}

\begin{flashcard}[File]{Was ist die Aufgabe von \path{/etc/profile}?}
	\begin{description}
		\item Konfigurationsdatei für Umgebungsvariabeln und erste \texttt{PATH}-Anweisung.
		
		\item Wird bei der Anmeldung eines Benutzers gelesen \textrightarrow Bei Änderungen Neuanmeldung nötig
	\end{description}
\end{flashcard}

\begin{flashcard}[File]{Was ist die Aufgabe von \path{/etc/.bashrc}?}
	\begin{description}
		\item Konfigurationsdatei für systemweite Einstellungen, Umgebungsvariabeln Aliases und Funktionen.
		
		\item Wird beim Start jeder neuen Shell gelesen.
	\end{description}
\end{flashcard}

\begin{flashcard}[File]{Was ist die Aufgabe von \path{~/.bash_profile}?}
	\begin{description}
		\item Nicht immer vorhanden.
		
		\item Konfigurationsdatei für benutzerspezifische Umgebungsvariabeln, weitere \texttt{PATH}-Anweisungen und den zu verwendeten Standard-Editor.
		                                                                                                      
		\item Wird bei der Anmeldung eines Benutzers sofort nach \path{/etc/profile} gelesen \textrightarrow Bei Änderungen Neuanmeldung nötig 
	\end{description}
\end{flashcard}

\begin{flashcard}[File]{Was ist die Aufgabe von \path{~/.bash_login}?}
	\begin{description}
		\item Alternative zu \path{~/.bash_profile}. Wird nur gelesen wenn diese nicht vorhanden ist. Inhalt und Zweck sind in beiden Files gleich.
		
		\item Nicht immer vorhanden.
		
		\item Konfigurationsdatei für benutzerspezifische Umgebungsvariabeln, weitere \texttt{PATH}-Anweisungen und den zu verwendeten Standard-Editor.
		
		\item Wird bei der Anmeldung eines Benutzers sofort nach \path{/etc/profile} gelesen \textrightarrow Bei Änderungen Neuanmeldung nötig 
	\end{description}
\end{flashcard}

\begin{flashcard}[File]{Was ist die Aufgabe von \path{~/.profile}?}
	\begin{description}
		\item Ursprüngliche Konfigurationsdatei der Bash.
		
		\item Wird nur gelesen wenn weder \path{~/.bash_profile} noch \path{~/.bash_login} vorhanden sind. Inhalt und Zweck sind in allen drei Files gleich.
		
		\item Nicht immer vorhanden.
		
		\item Konfigurationsdatei für benutzerspezifische Umgebungsvariabeln, weitere \texttt{PATH}-Anweisungen und den zu verwendeten Standard-Editor.
		
		\item Wird bei der Anmeldung eines Benutzers sofort nach \path{/etc/profile} gelesen \textrightarrow Bei Änderungen Neuanmeldung nötig 
	\end{description}
\end{flashcard}

\begin{flashcard}[File]{Was ist die Aufgabe von \path{~/.bashrc}?}
	\begin{description}
		\item Ursprüngliche Konfigurationsdatei der Bash.
		
		\item Wird beim Start jeder neuen Shell gelesen.
		
		\item Konfigurationsdatei für Aliases und Funktionen.
	\end{description}
\end{flashcard}

\begin{flashcard}[File]{Was ist die Aufgabe von \path{~/.bash_logout}?}
	\begin{description}
		\item Optionale Datei, die bei der Abmeldung eines Benutzers gelesen werden.
		
		\item Z.B. könnte sie den Monitor löschen.
	\end{description}
\end{flashcard}

\begin{flashcard}[Command]{Wie können die gesetzten Shellvariabeln angezeigt werden?}
	\texttt{set}
	
	\begin{description}
		\item \textbf{Options}
		
		\begin{description}
			\item \todo{options here}
		\end{description}
	\end{description}
	
	\lstinputlisting[style=bash, title=Example]{./src/set.sh}
\end{flashcard}

\begin{flashcard}[Command]{Wie kann in der Kommandozeile eine Variable deklariert werden?}
	\lstinputlisting[style=bash, title=Example]{./src/shellVar.sh}
	
	Wie im Beispiel sieht wird die Variable nicht automatisch an Subshells vererbt. Um die Variable zu vererben, muss vor dem aufruf der Bash das Kommando \texttt{export x} ausgeführt werden.
\end{flashcard}

\begin{flashcard}[Command]{Wie kann eine Variable Exportiert werden?}
	\lstinputlisting[style=bash, title=Example]{./src/export.sh}
\end{flashcard}

\begin{flashcard}[TO DO]{\todo{AB SEITE 92}}	
	\todo{AB SEITE 92}
\end{flashcard}